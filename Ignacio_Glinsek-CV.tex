%% start of file `template.tex'.
%% Copyright 2006-2015 Xavier Danaux (xdanaux@gmail.com).
%
% This work may be distributed and/or modified under the
% conditions of the LaTeX Project Public License version 1.3c,
% available at http://www.latex-project.org/lppl/.


\documentclass[11pt,a4paper,danish]{moderncv}        % possible options include font size ('10pt', '11pt' and '12pt'), paper size ('a4paper', 'letterpaper', 'a5paper', 'legalpaper', 'executivepaper' and 'landscape') and font family ('sans' and 'roman')
\usepackage[urw-garamond]{mathdesign}           % use garamond fonts
\usepackage[T1]{fontenc}                        % use garamond font encoding
\usepackage[english]{babel}

% moderncv themes
\moderncvstyle{banking}                             % style options are 'casual' (default), 'classic', 'banking', 'oldstyle' and 'fancy'
\moderncvcolor{black}                               % color options 'black', 'blue' (default), 'burgundy', 'green', 'grey', 'orange', 'purple' and 'red'
\nopagenumbers{}                                  % uncomment to suppress automatic page numbering for CVs longer than one page

% character encoding
\usepackage[utf8]{inputenc}                       % if you are not using xelatex ou lualatex, replace by the encoding you are using

% adjust the page margins
\usepackage[scale=0.85]{geometry}

\usepackage[unicode]{hyperref}                  % to use hyperlinks
\usepackage{xcolor}                             % better colors
\usepackage{color}                              % syntactic sugar

% personal data
\name{Roberto Ignacio}{Glinsek}
\phone[mobile]{+54~(911)~6743~5382}                   
\email{iglinsek@gmail.com}                              
\homepage{nachoglinsek.me}                         
\social[linkedin]{ignacioglinsek}                    
\social[github]{nakioman}                            
\extrainfo{Argentinian, Slovenian}

%----------------------------------------------------------------------------------
%            content
%----------------------------------------------------------------------------------
\begin{document}

%-----       resume       ---------------------------------------------------------
\makecvtitle

\section{Summary}
\begin{itemize}
  \item	15 years of software development experience.
  \item Excellent programming skills in \textbf{.NET}, \textbf{TypeScript} and \textbf{JavaScript},
        knowledgeable in \textbf{Python}, \textbf{Node} and \textbf{Docker}.
  \item	Experienced in \textbf{distributed applications}, \textbf{multi-threading} and \textbf{microservices}.
  \item	Strong technical, analytical and decision-making skills for problem identification and resolution.
  \item	Demonstrated technical leadership, strong interpersonal and communication skills.
  \item	Highly motivated and self-directed with the ability to work independently and as part of a team on multiple projects simultaneously.
  \item	Good knowledge of \textbf{new technologies} in general.
  \item Proven experience in \textbf{AWS} and \textbf{Azure} cloud services.

\end{itemize}

\section{Professional Experience}
\cventry{10/2017 -- Present}{Lead Application Engineer}{Rise Interactive}{}{}{Maintein and develop new features for the
  \httplink[Connex Platform]{https://www.riseinteractive.com/connex} as well as mentor developers into new technologies, my achievements include:
  \begin{itemize}
    \item	Migrate Microsoft MVC application to SPA using React, Redux and Sagas
    \item	Migrate Web Api using .NET Framework to .NET Core
    \item	Start using Docker containers to deploy our platform to AWS
    \item	Set and follow coding standards and best practices
  \end{itemize}}
\cventry{03/2017 -- 10/2017}{Full Stack Senior Developer}{SouthWorks}{}{}{Worked as a Microsoft vendor in different Azure projects:
  \begin{itemize}
    \item Maintein Azure Media Services portal
    \item Implement new features in the Azure Media Services portal
    \item Build from the ground up a video editor in ReactJS for the portal.
  \end{itemize}}
\cventry{06/2009 -- 03/2017}{Technical Leader}{AXA Assistance}{}{}{I was in charge of design, develop and lead a group of people that implement several in-house applications, my achievements included:
  \begin{itemize}
    \item Migrate our source control history from TFS to Git without loosing any commit in the history.
    \item Implement a software solution to automate the car rental and towing services manual operations, allowing the company to deliver faster operations at half the cost.
    \item Migrate our software from Windows Server 2003 to Windows Server 2012 R2.
    \item Started migrating our in house infrastructure to SaaS in Azure.
    \item Give in house talks throught the company about new technologies.
  \end{itemize}}
\cventry{04/2008 -- 05/2009}{Senior Software Developer}{Globant}{}{}{I worked as a consultant in various projects using mainly .NET, JavaScript and FLEX}
\cventry{04/2007 -- 04/2008}{Software Developer}{Tata Conltancy Services}{}{}{I worked on different .NET projects using ASP.NET techonologies}

\section{Studies}
\cventry{2001}{Computer Technician applied to Industrial production processes}{Maria Reina Institute}{Buenos Aires, Argentina}{}{}
\cventry{2004 -- 2010}{Computer Science}{University of Buenos Aires}{Buenos Aires, Argentina}{}{}

\section{Certifications}
\cventry{2013}{MS-10776 -- Developing Microsoft SQL Server 2012 Databases}{EXO Training}{Buenos Aires, Argentina}{}{}
\cventry{2012}{MS2933 -- Developing Business Process and Integration Solutions Using BizTalk}{EXO Training}{Buenos Aires, Argentina}{}{}
\cventry{2012}{ITIL Foundation Certificate in IT Service Management}{ITIL}{Buenos Aires, Argentina}{}{Registration number: 4638218}
\cventry{2009}{Microsoft Certified Professional}{Microsoft}{Buenos Aires, Argentina}{}{\httplink[Transcript ID: 877570, Access Code: IgnacioGlinsek]{https://mcp.microsoft.com/authenticate/validatemcp.aspx}}
\cventry{2006}{Security Seminar in Web applications and servers}{ArCERT}{Buenos Aires, Argentina}{}{}

\section{Languages}
\cvitem{English}{Fluent}
\cvitem{Spanish}{Native}

\section{Interests}
\cvlistitem{Gaming}
\cvlistitem{Reading}
\cvlistitem{Travel}

\end{document}
%% end of file `template.tex'.
